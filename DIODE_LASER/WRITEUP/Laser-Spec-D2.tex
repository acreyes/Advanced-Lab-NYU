% Lab Writeup for Diode-Laser Spectroscopy Lab
% Adam Reyes, George Wong
% Advanced Lab FAll 2013

%edited by adam to include methods on interferometer theory and
%discussion of broadened line spacings


\documentclass[paper=a4, fontsize=11pt]{scrartcl} % A4 paper and 11pt font size
\usepackage[left=2.5cm,top=2.5cm,right=2.5cm,bottom=2.5cm]{geometry} 
\usepackage{amsmath}
\usepackage{graphicx}
\usepackage{sectsty}
\usepackage{fancyhdr}
\pagestyle{fancyplain}
\usepackage{subcaption}
\usepackage{wrapfig}
\usepackage[english]{babel}

\numberwithin{equation}{section}
\numberwithin{figure}{section} 
\numberwithin{table}{section}
%\setlength\parindent{0pt}

\fancyhead[R]{\thepage} 
\fancyhead[L]{Reyes, Wong} 
\fancyhead[C]{Saturated Absorption Spectroscopy} 
\fancyfoot[L]{} 
\fancyfoot[C]{} 
\fancyfoot[R]{} 

\newcommand{\horrule}[1]{\rule{\linewidth}{#1}}

\title{	
Saturated Absorption Spectroscopy of Rubidium
\horrule{0.5pt}
\normalfont \normalsize 
\textsc{Advanced Experimental Physics }
}

\author{Adam Reyes \\ George Wong} % Your name

\date{\normalsize\today} % Today's date or a custom date


\begin{document}
\maketitle
%%%%%%%%%Abstract%%%%%%%%%%
\noindent\textbf{Abstract:}
A diode-laser swept through the absorption bands of Rubidium was used
to measure the absorption spectrum of the Rubidium lines. With the use
of a pump beam, the saturated absorption of Rubidium was found and
used to investigate the fine structure of Rubidium. 


\section{Background}


\indent It is well understood that an electron bound to a nucleus in some atom is restricted to a set of allowed energies, of which it can be found in. These energies correspond to ``wavefunctions'', which are eigenfunctions of the atom's energy Hamiltonian operator, that describe the probability of finding that electron at a given point in space. Because an electron is restricted to these allowed energies, it can absorb energy only of an amount that is exactly the difference to some other allowed energy, $\Delta E$. 

A photon of light carries some energy that corresponds to its frequency, given by: $E_\gamma = h\nu$, where $h$ is Planck's constant. If a photon carries energy $E_\gamma = \Delta E$, the energy of an allowed transition for an atomic electron, then the electron can absorb the photon and transition to an excited state. Since there is not a continuous spectrum of allowed energies the electron can only absorb photons for specific frequencies. We now consider some source of light, of varying frequency, is incident on a sample of some atom and the intensity of the light is measured on the other side of the sample. Since the atoms can only absorb particular frequencies of light, and transmit all others, in a plot of the measured transmitted intensity against light frequency, we would see dips in the transmitted intensities at the frequencies that correspond precisely to the allowed energy transitions of the atomic electrons. 





\section{Michelson Interferometer}

\section{Doppler Broadened Absorption}

\section{Saturated Absorption}



\begin{thebibliography}{99}
\bibitem{vanier}Jacques Vanier. Relaxation in rubidium-87 and the
  rubidium maser. Phys. Rev., 168:129-149, 1968.
\bibitem{harvard}"Rubidium Fluoresence." Cfa.harvard.edu. Harvard, n.d. Web.
\end{thebibliography}

\end{document}